\documentclass[a4paper,11pt]{ltjsarticle}
% 基本とドライバ関連
\usepackage{graphicx}
\usepackage{xcolor}
\usepackage{makeidx}
\usepackage{ascmac}

% LuaTeX-ja設定
\usepackage{luatexja}% 日本語したい
\usepackage[haranoaji,no-math,deluxe,expert,nfssonly,match,scale=1.0]{luatexja-preset}
\renewcommand{\kanjifamilydefault}{\gtdefault}% 既定をゴシック体に
\usepackage{lltjext}

% 数式系基本
\usepackage{amsmath}
\usepackage{amsthm}
\usepackage{amssymb}
\usepackage{mathtools}
% \mathtoolsset{showonlyrefs=true}
\usepackage{derivative}
\usepackage[b]{esvect}
\usepackage{nicematrix}
\usepackage{siunitx}
\usepackage{bm}

% 画像関係
\usepackage{animate}
\usepackage{svg}
\usepackage{tikz}

%表関連
\usepackage{multirow}

% 自然科学用追加
% \usepackage{chemmacros}
% \usechemmodule{all}
% \selectchemgreekmapping{fontspec}
\usepackage{chemfig}
\setchemfig{atom sep=1.5em}
% \ifdraft{}{\setchemfig{bond join=true}}

% 数式フォント設定
\usepackage{anyfontsize}
\newcommand{\sfscale}{0.98}
\newcommand{\ttscale}{0.96}
% \usepackage[mathnoalias]{iwona}
% % \setmainfont{Iwona}
% \usepackage[scale=\sfscale]{roboto}
% \usepackage[scale=\ttscale]{roboto-mono}
% \usepackage{BOONDOX-uprscr}
% \usepackage{BOONDOX-ds}

% ページ設定
\usepackage{geometry}
\geometry{left=25truemm, right=25truemm, top=25truemm, bottom=25truemm}
% \pagestyle{empty}

% hyperref関連
\usepackage{bookmark}
\usepackage{xurl}
\hypersetup{unicode,bookmarksnumbered=true,colorlinks=true,final}

%%%%%%%%%%%%%%%%%%%%%%%
\graphicspath{{../figure/}{../../figure/}}

\begin{document}

高分子を用いた材料は、金属や木材などの古くから使われてきている材料とは異なる特性を有する機能性材料として各種分野で大量に使用されています。
そして、今後もますますその使用量および適応分野は増加していくものと期待されています。

このメモは、おもに有機化学をベースにした合成化学的なアプローチで高分子を取り扱ってきたような化学系研究者を対象として作成しました。
すなわち、有機合成的にモノマーの化学構造を意識して、重合反応により生成したポリマーをモノマーユニットの繰り返した化合物として取り扱い、その物性については他の物性系の研究者と共同研究することで丸投げしてしまいがちな方たちに向けてまとめたわけです。

高分子を材料として使いこなしていくためには、高分子を化学的に合成するだけではなく、その特徴や性質を高分子物理という考え方で整理していく物理的なアプローチも必要となります。
しかしながら、高分子物理のテキストは物理系の学生を対象としている場合が多く、モノマーの化学構造のようなミクロな描写を無視した粗視化した表現として、高分子鎖を紐のようなものとして取り扱います。
そして、そのような紐の一般的な性質について議論を進めていくために、最初の数ページから、数学、統計、熱力学、統計力学等の基礎知識が必要となるような記述が出てくることがよくあります。
数学の得意でない化学系の学生は、このイントロの時点で躓いてしまうわけです。

このメモは、高分子物理の内容をきちんと理解できることを目的として、高分子物理のはじめの一歩ぐらいに対応する非常に初歩的な事項である「粗視化した高分子モデル」について、ゆっくりと理解できるように意図しています。
まず、高分子材料の機能を発現させる際に、ミクロとマクロをつなぐメゾスケールが重要であることを確認します。
その認識に立って、化学者になじみやすいミクロな化学構造を用いて高分子鎖の細長い紐のようなイメージを概観した後に、各種の高分子モデルについて示します。
最後に、粗視化したメゾスケールでの高分子の大きさについての議論を行います。
%最後に、高分子の特徴的なふるまいについて統計力学的な見方で理解を深めたいと考えています。

なお、数学アレルギーの方を考慮して、できるだけ本文中では数学的な表現を少なくしました。
そして、数式を取り入れた細かい説明は付録にまとめました。

残念ながら、この程度の知識を付けたからといってすぐに役立つわけではないのですが、これまでに学んできた「化学の見かた」に「物理のやり方」を加えていくための最初の一歩となればよいなと期待しております。
\end{document}