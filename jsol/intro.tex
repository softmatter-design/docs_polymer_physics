\documentclass[a4paper,11pt]{jlreq}
% 基本とドライバ関連
\usepackage{graphicx}
\usepackage{xcolor}
\usepackage{makeidx}
\usepackage{ascmac}

% LuaTeX-ja設定
\usepackage{luatexja}% 日本語したい
\usepackage[haranoaji,no-math,deluxe,expert,nfssonly,match,scale=1.0]{luatexja-preset}
\renewcommand{\kanjifamilydefault}{\gtdefault}% 既定をゴシック体に
\usepackage{lltjext}

% 数式系基本
\usepackage{amsmath}
\usepackage{amsthm}
\usepackage{amssymb}
\usepackage{mathtools}
% \mathtoolsset{showonlyrefs=true}
\usepackage{derivative}
\usepackage[b]{esvect}
\usepackage{nicematrix}
\usepackage{siunitx}
\usepackage{bm}

% 画像関係
\usepackage{animate}
\usepackage{svg}
\usepackage{tikz}

%表関連
\usepackage{multirow}

% 自然科学用追加
% \usepackage{chemmacros}
% \usechemmodule{all}
% \selectchemgreekmapping{fontspec}
\usepackage{chemfig}
\setchemfig{atom sep=1.5em}
% \ifdraft{}{\setchemfig{bond join=true}}

% 数式フォント設定
\usepackage{anyfontsize}
\newcommand{\sfscale}{0.98}
\newcommand{\ttscale}{0.96}
% \usepackage[mathnoalias]{iwona}
% % \setmainfont{Iwona}
% \usepackage[scale=\sfscale]{roboto}
% \usepackage[scale=\ttscale]{roboto-mono}
% \usepackage{BOONDOX-uprscr}
% \usepackage{BOONDOX-ds}

% ページ設定
\usepackage{geometry}
\geometry{left=25truemm, right=25truemm, top=25truemm, bottom=25truemm}
% \pagestyle{empty}

% hyperref関連
\usepackage{bookmark}
\usepackage{xurl}
\hypersetup{unicode,bookmarksnumbered=true,colorlinks=true,final}

%%%%%%%%%%%%%%%%%%%%%%%
\graphicspath{{../figure/}{../../figure/}}

\begin{document}

\section{高分子の材料設計}

\subsection{高分子材料の機能設計}
高分子を用いた材料は、金属や木材などの古くから使われてきている材料とは異なる特性を有する機能性材料として各種分野で大量に使用されています。
そして、今後もますますその使用量および適応分野は増加していくものと期待されています。

高分子の特性を活かして機能を設計していくために、材料設計という視点から高分子材料を理解して使いこなしていく必要性も日増しに高まってきています。

\subsubsection{高分子特有の難しさ}
しかしながら、高分子という名前に現れているように、分子量が大きいということに起因して低分子材料の機能設計とは異なるアプローチが必要となります\footnote{
    高分子とは、モノマーが連なって結合して形成されるヒモのようなものであるため、その長さを正確にコントロールすることは困難で、大抵の場合に連鎖の長さの分布が生じます。
さらに、高分子の化学構造を考える場合には、立体配置(コンフィギュレーション)と立体配座(コンフォメーション)という、2つの異なった観点で考える必要があります。
}。

求められる条件を満たす材料を開発するためには、モノマーの選択のみならず重合プロセスに依存して材料特性も変化する場合が多いため、何度も研究と実験を繰り返す必要があり、ときには1つの材料開発に十年オーダーの長い期間を要することもありました。

\subsubsection{シミュレーションの活用}
高分子材料の研究開発においては、以前から分子動力学法によるもの等の様々なシミュレーションを活用した検討が進められてきました。
とりわけ、近年の並列コンピューティングの急速な進行によりコンピュータ資源の処理速度の大幅な向上にも助けられ、大規模シミュレーションも当たり前に行われるようになってきています。

しかしながら、シミュレーションの検討においても、上述の高分子量ということに付随した多くの困難が生じる場合も多く、単純に材料設計へと展開することは困難でした。
また、状況によっては研究のための検討に偏りすぎていたりして、マクロな材料特性(例えば力学特性等)を手軽に検討することは達成できているとは言い難い状況でした。

\subsection{マテリアルズ・インフォマティクス}
近年はマテリアルズ・インフォマティクス(MI)という概念のもとに、機械学習などの情報科学(インフォマティクス)を用いて様々な材料開発の効率が高められようとしています。
AIなどのデジタル技術の発展に伴い、膨大な実験や論文のデータを解析できるようになったことが大きなポイントとされています。

\subsubsection{化学製品におけるMI}
求める性能を満たす材料の組み合わせや、製造方法を予測できるなどの効果が得られるという謳い文句にのり、化学薬品や石油などを原料として製品を製造するプロセス系製造業での応用が拡大しつつあります。
この活用により、研究者の経験や知識、スキルに頼っていたこれまでの材料開発の検討プロセスが大幅に単純化できるとされています。

\subsubsection{高分子材料へのMI適応の問題点}
MIでは、過去の論文データを機械学習で分析させ素材の組み合わせを導き出したりすることが可能とされていますが、高分子材料への適応においては、この過去のデータで苦労している場合も多く見受けられます。

計算機科学においては、以前から"Garbage In, Garbage Out"という言葉がよく知られていました\footnote{
    \href{https://ja.wikipedia.org/wiki/Garbage_in,_garbage_out}{Wikipedia}によれば、 「"Garbage in, gospel out"という文字列として用いられることがある。これはコンピュータがどのような性向をもって処理しているのかを知らずにコンピュータのデータを過度に信仰することに対する皮肉」とのこと。
}。
高分子材料のMIにおいては、良質な実験データを比較しうる状態で検討することが重要ですが、過去のデータの採集状況が明確でなかったり、サンプルの分子量等の詳細な情報が不十分だったりする場合が散見されます。

\subsection{MIへの添加物という提案}
MIの持つ強みを最大限に活かしながら、高分子材料への適応を上手に進めていくためには、さらなる方法論を添加していくことが有効だと考えています。
\subsubsection{MI+シミュレーション}
過去の実験データの使いにくさを補う方法としてシミュレーションの有効活用も提案されています。
すなわち、過去のデータの曖昧さを解消するために、Full-Atomistic MDシミュレーションにより適切なパラメタを策定することで補完するデータを取得する方法です。
このようなアプローチの有効性も近年広まってきているようです。

\subsubsection{物理的な思考のおすすめ}
最近の方には笑われるような当方の古い感覚なのでしょうが、開発の方向性の判断には何らかの経験に基づく知性が必須なのではと思っています。
多変量解析的な感覚で「強い相関が見られるパラメタであれば有効であろうと判断」することを一概に間違いというわけではないのですが、その有効性の裏に潜むはずの物理を想像できればさらに開発のディーテールが深まると感じています\footnote{
    「神は細部に宿る」とも言いますし、あんまり荒っぽい進め方は結局二度手間になることが多いと感じたことは数え切れません。
}。

具体的には、物理化学に分類されるような基礎的な事項をベースとして、高分子物理と呼ばれるようなジャンルの知識をきちんと理解しておくことが、結局はムダの少ない開発への早道なのだと強く思っています。

\subsection{高分子物理と材料設計}
具体的な例を上げて説明しましょう。

\subsubsection{高分子の広がりと破壊挙動の関係}
高分子鎖の広がりを「特性比」というパラメタで表すことで、実在高分子の破壊挙動の振る舞いとの間に強い相関を見出すことができるという報告がWuによって報告されています。
この論文自体は、かなり突っ込みどころが満載なのですが、基本的な感覚はそれほど間違っているとは思えません。
この論文の後に提案された"Packing Length"という概念を用いて、Fettersらがイメージしやすい物理的な意味をうまく説明しており、彼らの仮説としては「絡み合い点間分子量」が支配的な因子ではないかと提案しています。

このような検討をトレースしようとする場合に、単純に「特性比」というパラメタや算出可能な"Packing Length"を単なる数値パラメタとして用いる事自体が悪いわけではありません。
しかしながら、その物理的な意味や導出過程までも理解していると、検討過程でなにか不具合があったときもその原因の推定が容易になります。
そのためには、抽象化した理想鎖の概念から排除体積効果のある実在鎖までのロジックの流れを理解していると便利になるでしょう。

\subsubsection{高分子の相溶性と相互作用パラメタ}
高分子の相溶性や固体表面への濡れ性ということも、材料開発においては重要なパラメタとなります。
これらの特性が、相互作用を表す$\chi$パラメタという因子で影響されるということは、かなり多くの人に周知されているでしょう。
しかしながら、分子量$N$との積の形$\chi N$が無次元としての普遍的な物理量であることや、ブロックポリマーの構造形成などもモデル的な意味ではきれいに説明できることについての理解はかなり限られるでしょう。

高分子同士の相互作用については、かなり簡略化したモデルであるFlory-Hugginsの格子モデルを理解することで概略のイメージを持つことができます。
その中で、セグメント間相互作用というエンタルピックな挙動とポリマー鎖の並進エントロピーとの取り合いという自由エネルギーとしての振る舞いを理解できます。
そして、その演繹として高分子鎖の形態エントロピーのことも考慮できれば、壁近傍での高分子鎖の枯渇現象も理解でき、粘着剤等の応用検討にも役立ちます。



% \begin{table}[tb]
%     \centering
%         \begin{tabular}{c|c|c|c} \hline
%             \multirow{2}{*}{物質}&引っ張り強度&密度&比強度\\
%                 &\unit{MPa}&\unit{g/cm^3}&\unit{kN.m/kg}\\ \hline \hline
%             コンクリート&10&2.30&4.4\\ \hline
%             ゴム&15&0.92&16\\ \hline
%             SUS304&505&8.00&63\\ \hline
% % ナイロン	78	1.13	69.0	7.04	[4]
% % ポリプロピレン	80	0.90	88.88	9.06	[6]
% % チタン合金	1250	4.61	260	27.65	[8]
% % マグネシウム合金	510	1.86	274	27.87	[9]
% % CFRP	1240	1.58	785	80.0	[12]
% % 蜘蛛の糸	1400	1.31	1069	109	
% % ケブラー	3620	1.44	2514\\
%         \end{tabular}
% \end{table}


\end{document}