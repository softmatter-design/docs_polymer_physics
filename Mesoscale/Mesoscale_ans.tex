\documentclass[a4paper,11pt]{jlreq}
% 基本とドライバ関連
\usepackage{graphicx}
\usepackage{xcolor}
\usepackage{makeidx}
\usepackage{ascmac}

% LuaTeX-ja設定
\usepackage{luatexja}% 日本語したい
\usepackage[haranoaji,no-math,deluxe,expert,nfssonly,match,scale=1.0]{luatexja-preset}
\renewcommand{\kanjifamilydefault}{\gtdefault}% 既定をゴシック体に
\usepackage{lltjext}

% 数式系基本
\usepackage{amsmath}
\usepackage{amsthm}
\usepackage{amssymb}
\usepackage{mathtools}
\mathtoolsset{showonlyrefs=true}
\usepackage{derivative}
\usepackage[b]{esvect}
\usepackage{nicematrix}
\usepackage{siunitx}

% 画像関係
% \usepackage{animate}
\usepackage{svg}
\usepackage{tikz}

%表関連
\usepackage{multirow}

% 自然科学用追加
% \usepackage{chemmacros}
% \usechemmodule{all}
% \selectchemgreekmapping{fontspec}
\usepackage{chemfig}
% \ifdraft{}{\setchemfig{bond join=true}}
% 数式フォント設定
\newcommand{\sfscale}{0.98}
\newcommand{\ttscale}{0.96}
% \usepackage[mathnoalias]{iwona}
% \setmainfont{Iwona}
% \usepackage[scale=\sfscale]{roboto}
% \usepackage[scale=\ttscale]{roboto-mono}
% \usepackage{BOONDOX-uprscr}
% \usepackage{BOONDOX-ds}

% ページ設定
\usepackage{geometry}
\geometry{left=25truemm, right=25truemm, top=25truemm, bottom=25truemm}
% \pagestyle{empty}

% hyperref関連
\usepackage{bookmark}
\usepackage{xurl}
\hypersetup{unicode,bookmarksnumbered=true,hidelinks,final}

%%%%%%%%%%%%%%%%%%%%%%%
\graphicspath{{../figure/}}

\begin{document}
\subsection{メゾスケールと高分子の形}
\begin{enumerate}
\item (メゾスケール)\\メゾスケールのことについて、できるだけ自分の言葉で説明してみてください。\\
		\textbf{(解答例)}\\メソあるいはメゾとは、「中間」という意味を表す接頭語です。
		したがって、高分子化学の議論においてはメゾスケールという言葉の示す大きさは、以下の2つの間の大きさを指すことになり、長さの次元で\num{1e6}にも及ぶスケール領域ということになる。
        \begin{itemize}
            \item 有機化学で対象とする炭素結合の長さ程度の、たかだか\qty{1}{nm}程度の大きさのミクロなスケール
            \item 我々が目に見えて実感できるような\qty{1}{mm}程度以上の大きさ(マクロスケール)
        \end{itemize}
		材料の力学特性のような機能は、一般に、マクロスケールで発現するものです。
        したがって、マクロで機能を有する材料を設計する際には、化学的な感覚でナノスケールの大きさで設計することと実際に評価しているマクロスケールとの間のメゾスケールが重要になります。
		高分子の場合、その鎖としての構造の自由度が非常に高く、低分子には見られない特有の形状を取ることができる。
		また、各種の構造間の遷移に伴う活性化エネルギーが熱エネルギー程度($\simeq k_B T$)であり、構造変換が容易に起こる。
		このようなことから、高分子材料においては、マクロな機能性へのメゾスケール構造の寄与が非常に大きいことが知られている。

\item (高分子の形)\\高分子の形について、立体配置(コンフィギュレーション)と立体配座(コンフォメーション)という言葉を意識して、説明してください。\\
		\textbf{(解答例)}\\立体配置は、モノマー内、あるいは、モノマー間の共有結合の生成により規定される構造の様式です。
        このなかには、頭-尾結合、幾何異性体、立体規則性等のモノマー単位の構造から、高分子鎖の主鎖の分岐やデンドリマー、環状構造のような大きなスケールの構造も含まれている。
		構造の大きさに関わらず、共有結合を切断することなしには異なる構造へと遷移できないものであり、高分子の特性を決定する大きな要因です。

		一方、立体配座は高分子鎖中の炭素・炭素単結合周りの回転に基づくものであり、一般に熱揺らぎ程度で容易に遷移してしまうという特徴があり、化学構造式としては書き表すことが困難で実感しにくい。
		しかしながら、高分子の膨大な内部自由度の原因であり、ソフトマターとしての特徴的な機能発現に大きく寄与している。

\end{enumerate}
\end{document}