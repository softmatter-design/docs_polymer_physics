\documentclass[a4paper,11pt]{ltjsarticle}
% 基本とドライバ関連
\usepackage{graphicx}
\usepackage{xcolor}
\usepackage{makeidx}
\usepackage{ascmac}

% LuaTeX-ja設定
\usepackage{luatexja}% 日本語したい
\usepackage[haranoaji,no-math,deluxe,expert,nfssonly,match,scale=1.0]{luatexja-preset}
\renewcommand{\kanjifamilydefault}{\gtdefault}% 既定をゴシック体に
\usepackage{lltjext}

% 数式系基本
\usepackage{amsmath}
\usepackage{amsthm}
\usepackage{amssymb}
\usepackage{mathtools}
\mathtoolsset{showonlyrefs=true}
\usepackage{derivative}
\usepackage[b]{esvect}
\usepackage{nicematrix}
\usepackage{siunitx}

% 画像関係
% \usepackage{animate}
\usepackage{svg}
\usepackage{tikz}

%表関連
\usepackage{multirow}

% 自然科学用追加
% \usepackage{chemmacros}
% \usechemmodule{all}
% \selectchemgreekmapping{fontspec}
\usepackage{chemfig}
% \ifdraft{}{\setchemfig{bond join=true}}
% 数式フォント設定
\newcommand{\sfscale}{0.98}
\newcommand{\ttscale}{0.96}
% \usepackage[mathnoalias]{iwona}
% \setmainfont{Iwona}
% \usepackage[scale=\sfscale]{roboto}
% \usepackage[scale=\ttscale]{roboto-mono}
% \usepackage{BOONDOX-uprscr}
% \usepackage{BOONDOX-ds}

% ページ設定
\usepackage{geometry}
\geometry{left=25truemm, right=25truemm, top=25truemm, bottom=25truemm}
% \pagestyle{empty}

% hyperref関連
\usepackage{bookmark}
\usepackage{xurl}
\hypersetup{unicode,bookmarksnumbered=true,hidelinks,final}

%%%%%%%%%%%%%%%%%%%%%%%
\graphicspath{{../figure/}}

\begin{document}

まず、有機化学者が化学式を用いて構造を考えるような比較的分子量の低い分子(たとえば、モノマー単位に対応)の微小(ミクロ)な大きさのイメージと、実際に高分子材料を手に取って使用するような大きさ(マクロ)との間にある、中間的な大きさに対応する「メゾスケール」という考え方から始めていきましょう。
ここでは、この文章では大きさを測る基準となる物差しという意味でスケールという言葉を使って、対照としている物質の大きさを表すことにします。
その議論の後に、高分子の形についても簡単に触れます。

\subsection{ミクロとマクロ}

有機化学で対象とする炭素系の分子において基本的な長さとなる炭素結合は、1.54\AA $\simeq$ \qty{10e-1}{\nano\meter}程度の長さです。
したがって、化学式で全構造を書けるような低分子の化学構造をベースに考えているということは、たかだか 1 nm 程度の微小な大きさで物質を捉えていることになります。
実際、有機合成のような分野においては、このような考え方で十分な場合がほとんどなのでしょう。

原子をあらわに見るようなこの程度の大きさに関する議論をミクロスケールと呼びます。

一方、我々が目に見えて実感できるのは、\qty{1}{\mm}程度以上の大きさでの事象となっています。
たとえば、身の回りに多数存在するプラスティック材料の強さを表す力学特性のような理解しやすい機能というものは、一般に、手で触って実感できるような大きさで発現するものです。
このような大きさでの議論をマクロスケールと呼び、

したがって、材料の各種機能を調べて使いこなすということは、マクロな大きさでの評価を用いる場合が多いことになります。

このことを一般的な言葉にしてみれば、「木を見て森を見ず」\footnote{余談になりますが、宇宙、銀河の大きさから原子核の大きさに至るまでのとんでもないスケール(尺度)をなんとなくイメージできるサイトを株式会社 Nikon が作っています。
	\url{https://www.jp.nikon.com/company/corporate/sp/universcale/scale.html}
	これは一見の価値があると思います。
}

\subsection{メゾスケール}

したがって、マクロな大きさで機能を有する材料の設計を行う際には、化学的な感覚でナノスケールの大きさで設計することと、実際に評価しているマクロスケールとの間にある、長さの次元で\num{10e6}にも及ぶ「メゾ(中間的)スケール領域」をきちんと取り扱うことの、両方が重要になってくるわけです。
ただし、対象とする機能に応じて、このメゾスケールの振る舞いの重要性は異なってきます
\footnote
{
例えば、一般に光学特性のような分子構造に強く依存する特性はミクロな議論が主となる場合が多く、自動車タイヤの力学特性のようなマクロよりな特性の場合はメゾスケールでの階層構造が非常に重要であることが知られてきている。
}。

高分子の場合、このメモで示すように、その鎖としての構造の自由度が非常に高く、マクロな機能性へのメゾスケール構造の寄与が非常に大きいことが知られています。
このことが、低分子には見られない高分子特有の振る舞いの原因であり、このような高分子の形状に起因した特徴的な振る舞いをうまく利用して材料設計を行おうという考えが、近年、急速に広まってきています
\footnote
{
このような流れの端緒は、ノーベル賞受賞者であるド・ジャンによるものであり、金属のように硬い物質との対比で、「ソフトマター」と名付けられています。
ソフトマターの対象となるのは、高分子だけではなくて、界面活性剤が形成するミセルやビヒクルのようなものも含まれており、決して、分子量が大きいことが必須なわけではありません。

一般にソフトマターと呼ばれるものは、各種の構造間の遷移に伴う活性化エネルギーが熱エネルギー程度($\simeq k_B T$)であり、構造変換が容易に起こるような、ソフトな(移ろいやすい)物質あるいは事象を指しています。
}。
\subsection{高分子の形}

高分子の形を考える場合に、立体配置(コンフィギュレーション)と立体配座(コンフォメーション)という、2つの異なった観点で考える必要があります。

立体配置は、モノマー内あるいはモノマー間の共有結合の生成により規定される構造の様式であり、主として以下の3つのような結合様式に基づいています。
% \vspace{-2mm}
\begin{itemize}
\item 
頭-尾、頭-頭結合
\item
幾何異性体
\item
立体規則性(光学異性体)
\end{itemize}
% \vspace{-2mm}
さらには、高分子鎖の主鎖の分岐やデンドリマー、環状構造のような大きなスケールの構造も含まれてきます。
いづれにせよ、構造の大きさにかかわらず、共有結合を切断することなしには異なる構造へと遷移できないものである。
一般に、合成系研究者にとっての高分子の設計とは、この立体配座を主に考えている場合が多いようです。

一方、立体配座は、第二章に示すように、高分子鎖中の C-C 結合周りの回転に基づくものであり、一般に、熱揺らぎ程度で容易に遷移してしまうものである。
したがって、単一の構造として捉えることは困難となり、統計的な手法で評価する必要が生じることになる。

立体配置は、適切な記述方法を選択すれば、化学構造式としての認識も容易に行えるのであるが、立体配座のほうは、化学構造式としては書き表すことが困難であり、実感することが困難なものである。
しかしながら、後述するように、この立体配座こそが高分子の膨大な内部自由度を生み、高分子材料の特徴的な機能発現に大きく寄与するものである。

\subsection{章末問題}

	\begin{enumerate}
		\item
		(メゾスケール)\\
		メゾスケールのことについて、できるだけ自分の言葉で説明してみてください。
		\item
		(高分子の形)\\
		高分子の形について、立体配置(コンフィギュレーション)と立体配座(コンフォメーション)という言葉を意識して、説明してください。
%		(ヒント)\\
%		C-C 結合の結合長が 1.54 \AA、三つの炭素が形成する C-C-C 結合が約 $109.5^o$ の結合角ということを考慮して、平面ジグザグ構造でのモノマー二つ分の絵を書いてみれば、
%		ポリマー鎖の伸長方向とそれに垂直な方向との長さが見積れます。

	\end{enumerate}

\end{document}