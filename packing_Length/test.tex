\documentclass[a4paper,11pt]{ltjsarticle}
% 基本とドライバ関連
\usepackage{graphicx}
\usepackage{xcolor}
\usepackage{makeidx}
\usepackage{ascmac}

% LuaTeX-ja設定
\usepackage{luatexja}% 日本語したい
\usepackage[haranoaji,no-math,deluxe,expert,nfssonly,match,scale=1.0]{luatexja-preset}
\renewcommand{\kanjifamilydefault}{\gtdefault}% 既定をゴシック体に
\usepackage{lltjext}

% 数式系基本
\usepackage{amsmath}
\usepackage{amsthm}
\usepackage{amssymb}
\usepackage{mathtools}
\mathtoolsset{showonlyrefs=true}
\usepackage{derivative}
\usepackage[b]{esvect}
\usepackage{nicematrix}
\usepackage{siunitx}

% 画像関係
% \usepackage{animate}
\usepackage{svg}
\usepackage{tikz}

%表関連
\usepackage{multirow}

% 自然科学用追加
% \usepackage{chemmacros}
% \usechemmodule{all}
% \selectchemgreekmapping{fontspec}
\usepackage{chemfig}
% \ifdraft{}{\setchemfig{bond join=true}}
% 数式フォント設定
\newcommand{\sfscale}{0.98}
\newcommand{\ttscale}{0.96}
% \usepackage[mathnoalias]{iwona}
% \setmainfont{Iwona}
% \usepackage[scale=\sfscale]{roboto}
% \usepackage[scale=\ttscale]{roboto-mono}
% \usepackage{BOONDOX-uprscr}
% \usepackage{BOONDOX-ds}

% ページ設定
\usepackage{geometry}
\geometry{left=25truemm, right=25truemm, top=25truemm, bottom=25truemm}
% \pagestyle{empty}

% hyperref関連
\usepackage{bookmark}
\usepackage{xurl}
\hypersetup{unicode,bookmarksnumbered=true,hidelinks,final}

%%%%%%%%%%%%%%%%%%%%%%%
\graphicspath{{../figure/}}

\begin{document}

あああ

The Packing Length 
In effect, p can be likened to the molecular diameter of the repeat unit in a polymer chain. A mental picture of the meaning of p can be gained in the following way. The packing length, as noted, is defined as 

\begin{equation}
    p-\dfrac{M}{\langle R^2 \rangle_0 \rho N_a}
\end{equation}
Equation (1A) can also be expressed in terms of Flory’s characteristic ratio as 

The average volume of a chain per bond is given as 

Thus, the packing length can be expressed as 

The parameter V 
, where h represents the diameter of a cylinder of length  
0 
swept out 
by the chain repeat unit. 

Thus, 

where the product C 
 
 
0 
is the Kuhn step length 
l. One can thus think of a chain as being made up of freely jointed rods of length l = C and diameter h. The packing length is thus proportional to the ratio of the cross-sectional area to the length of each rod. Hence, the smaller h is with respect to the skinnier the chain and the smaller p and Vb 

\end{document}