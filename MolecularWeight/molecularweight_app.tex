\documentclass[a4paper,11pt]{jlreq}
% 基本とドライバ関連
\usepackage{graphicx}
\usepackage{xcolor}
\usepackage{makeidx}
\usepackage{ascmac}

% LuaTeX-ja設定
\usepackage{luatexja}% 日本語したい
\usepackage[haranoaji,no-math,deluxe,expert,nfssonly,match,scale=1.0]{luatexja-preset}
\renewcommand{\kanjifamilydefault}{\gtdefault}% 既定をゴシック体に
\usepackage{lltjext}

% 数式系基本
\usepackage{amsmath}
\usepackage{amsthm}
\usepackage{amssymb}
\usepackage{mathtools}
% \mathtoolsset{showonlyrefs=true}
\usepackage{derivative}
\usepackage[b]{esvect}
\usepackage{nicematrix}
\usepackage{siunitx}
\usepackage{bm}

% 画像関係
\usepackage{animate}
\usepackage{svg}
\usepackage{tikz}

%表関連
\usepackage{multirow}

% 自然科学用追加
% \usepackage{chemmacros}
% \usechemmodule{all}
% \selectchemgreekmapping{fontspec}
\usepackage{chemfig}
\setchemfig{atom sep=1.5em}
% \ifdraft{}{\setchemfig{bond join=true}}

% 数式フォント設定
\usepackage{anyfontsize}
\newcommand{\sfscale}{0.98}
\newcommand{\ttscale}{0.96}
% \usepackage[mathnoalias]{iwona}
% % \setmainfont{Iwona}
% \usepackage[scale=\sfscale]{roboto}
% \usepackage[scale=\ttscale]{roboto-mono}
% \usepackage{BOONDOX-uprscr}
% \usepackage{BOONDOX-ds}

% ページ設定
\usepackage{geometry}
\geometry{left=25truemm, right=25truemm, top=25truemm, bottom=25truemm}
% \pagestyle{empty}

% hyperref関連
\usepackage{bookmark}
\usepackage{xurl}
\hypersetup{unicode,bookmarksnumbered=true,colorlinks=true,final}

%%%%%%%%%%%%%%%%%%%%%%%
\graphicspath{{../figure/}{../../figure/}}

\usepackage{xr}
\externaldocument{./molecularweight}

\begin{document}
\subsection{分子量分布の分散}
数微分分布関数 $f_n(M)$ の標準偏差は、下式の一行目で定義され、以下のように展開できる。
\begin{align}
s_n^2
	&= \displaystyle \int_0^{\infty}(M - \bar{M}_n)^2 f_n(M) \odif{M} \notag \\
	&= \displaystyle \int_0^{\infty}(M^2 - 2M \bar{M_n} + \bar{M}_n^2) f_n(M) \odif{M} \notag \\
	&= \displaystyle \int_0^{\infty} M^2 f_n(M) \odif{M} - 2 \bar{M}_n \displaystyle \int_0^{\infty} M f_n(M) \odif{M} + \bar{M}_n^2 \displaystyle \int_0^{\infty} f_n(M) \odif{M} \notag \\
	&= \displaystyle \int_0^{\infty} M^2 f_n(M) \odif{M} - 2 \bar{M}_n \bar{M}_n + \bar{M}_n^2  \notag \\
	&= \displaystyle \int_0^{\infty} M^2 f_n(M) \odif{M} - \bar{M}_n^2
\end{align}

なお、四行目への展開においては、$\displaystyle \int_0^{\infty} M f_n(M) \odif{M} = \bar{M}_n$ と $\displaystyle \int_0^{\infty} f_n(M) \odif{M} = 1$ を用いた。

ここで、(\ref{eq:Mw}) 式より、
\begin{align}
{\bar M}_w 
	&= \dfrac{\sum_i n_i M_i^2}{\sum_i n_i M_i}\notag \\[6pt]
	&= \dfrac{\sum_i n_i M_i^2}{\sum_i n_i} \times \dfrac{\sum_i n_i}{\sum_i n_i M_i}\notag \\[6pt]
	&= \displaystyle \int_0^{\infty} M^2 f_n(M) \odif{M} \times \dfrac{1}{\bar{M}_n }\notag \\[6pt]
\therefore \displaystyle \int_0^{\infty} M^2 f_n(M) \odif{M} &= \bar{M}_n \bar{M}_w
\end{align}
なお、三行目への展開では、数微分分布関数 $f_n(M)$ の二次のモーメントであることを用いている。

この結果を代入して、
\begin{align}
s_n^2 &= \bar{M}_n \bar{M}_w - \bar{M}_n^2\notag \\[6pt]
\therefore \dfrac{s_n}{\bar{M}_n} &= \left(\dfrac{\bar{M}_w}{ \bar{M}_n} - 1 \right)^{-1/2}
\end{align}
を得る。
なお、二行目へは、両辺を $\bar{M}_n^2$ で除した後に、平方根を取っている。
\end{document}