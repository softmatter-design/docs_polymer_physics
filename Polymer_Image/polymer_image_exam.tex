\documentclass[a4paper,11pt]{jlreq}
% 基本とドライバ関連
\usepackage{graphicx}
\usepackage{xcolor}
\usepackage{makeidx}
\usepackage{ascmac}

% LuaTeX-ja設定
\usepackage{luatexja}% 日本語したい
\usepackage[haranoaji,no-math,deluxe,expert,nfssonly,match,scale=1.0]{luatexja-preset}
\renewcommand{\kanjifamilydefault}{\gtdefault}% 既定をゴシック体に
\usepackage{lltjext}

% 数式系基本
\usepackage{amsmath}
\usepackage{amsthm}
\usepackage{amssymb}
\usepackage{mathtools}
\mathtoolsset{showonlyrefs=true}
\usepackage{derivative}
\usepackage[b]{esvect}
\usepackage{nicematrix}
\usepackage{siunitx}

% 画像関係
% \usepackage{animate}
\usepackage{svg}
\usepackage{tikz}

%表関連
\usepackage{multirow}

% 自然科学用追加
% \usepackage{chemmacros}
% \usechemmodule{all}
% \selectchemgreekmapping{fontspec}
\usepackage{chemfig}
% \ifdraft{}{\setchemfig{bond join=true}}
% 数式フォント設定
\newcommand{\sfscale}{0.98}
\newcommand{\ttscale}{0.96}
% \usepackage[mathnoalias]{iwona}
% \setmainfont{Iwona}
% \usepackage[scale=\sfscale]{roboto}
% \usepackage[scale=\ttscale]{roboto-mono}
% \usepackage{BOONDOX-uprscr}
% \usepackage{BOONDOX-ds}

% ページ設定
\usepackage{geometry}
\geometry{left=25truemm, right=25truemm, top=25truemm, bottom=25truemm}
% \pagestyle{empty}

% hyperref関連
\usepackage{bookmark}
\usepackage{xurl}
\hypersetup{unicode,bookmarksnumbered=true,hidelinks,final}

%%%%%%%%%%%%%%%%%%%%%%%
\graphicspath{{../figure/}}

\begin{document}
\subsection{章末問題}
	\begin{enumerate}
		\item
		\label{it:2-1}
		(モノマー単位の大きさの見積もり)\\
		文中に示したポリエチレンの平面ジグザグ構造に基づく伸び切り構造の大きさの見積もりを、具体的に説明してください。\\
		(ヒント)\\
		\chemfig{C-C}結合の結合長が\qty{0.154}{nm}、3つの炭素が形成する\chemfig{C-C-C}結合は約\ang{109.5}の結合角ということを考慮して、平面ジグザグ構造でのモノマー二つ分の絵を書いてみれば、
		ポリマー鎖の伸長方向とそれに垂直な方向との長さが見積れます。

		\item
		\label{it:2-2}
		(伸び切り鎖の長さの見積もり)\\
		ポリエチレンの分子量が\num{100000}であった場合に、その平面ジグザグ構造の伸び切り鎖の長さが約\qty{900}{nm}となることを、具体的に説明してください。\\
		(ヒント)\\
		分子量が\num{100000}であった場合に、その重合度がいくつであるかを算出すればいいだけです。

		\item
		\label{it:2-3}
		(出現確率の比)\\
		エネルギーの異なる二つの状態(エネルギー差$\Delta E$)の出現確率の比$r(\Delta E)$が以下の式となることを説明してください。
		\begin{equation*}
		r(\Delta E) = \exp(-\beta \Delta E)=\exp \left( -\dfrac{\Delta E}{k_B T} \right)
		\end{equation*}
		(ヒント)\\
		熱平衡状態であるカノニカル分布において、エネルギー準位が $E_i$ である状態 $i$ の出現確率 $P(E_i)$ は、下式で表されることを使って、
		二つのエネルギー状態の比を取ってください。\\
		なお、指数関数同士の割り算は、指数の引き算となります。
		\begin{align*}
		P(E_i) = \dfrac{1}{Z} \exp(-\beta E_i)
		\end{align*}

		\item
		(鎖の曲り方の具体例)\\
		以下の二つのクイズを考えてください。

		\begin{enumerate}
			\item
			\label{it:2-4}
			(ポリエチレンの例)\\
			室温(\qty{27}{\degreeCelsius})のポリエチレンで、トランスとゴーシュとのポテンシャル・エネルギー差 $\Delta \varepsilon$ が\qty{2.1}{kJ mol^{-1}}であったとした場合、
			トランス連鎖はどのぐらい続くと見積もればいいことになるでしょうか。\\
			(ヒント)\\
			(\ref{eq:ratio})式に、上記のエネルギー差を入れれば、それぞれの状態の出現確率の比が求まります。
			なお、$k_B \simeq$ \qty{1.4 E-23}{J K^{-1}}です。 

			\item
			\label{it:2-5}
			(持続長)\\
			このメモでは詳細に立ち入りませんが、高分子鎖の曲がりやすさを考慮したモデルとして、ミミズ鎖(Worm-like chain)と呼ばれるものがあります。
			このモデルにおいては、粗視化した単位として、モノマー単位が平面ジグザグ構造でまっすぐにつながる長さを統計的に見積もって、「持続長 $l_p$」と呼ぶものを使います。\\
			持続長は、以下の表式で表すことができます。
			\begin{equation*}
			l_p = l_0 \exp \left( - \dfrac{\Delta \varepsilon}{k_B T} \right)
			\end{equation*}
			ここで、$l_0$ はモノマー単位の長さであり、ポリエチレンの場合は、\chemfig{C-C}結合の結合長\qty{0.154}{nm}を用います。\\
			この式を用いて、上記のポリエチレンの場合の持続長を求めてください。\\
			(ヒント)\\
			結局、統計的に見てトランス連鎖が続く状態を考えて、その具体的な長さを見ているだけです。
	\end{enumerate}
	\item
	\label{it:2-6}
	(トランス・ゴーシュの遷移)\\
	ブタンにおける C-C 結合周りの回転振動数は、\qty{1E12}{sec^{-1}}のオーダーで生じるとして、トランス$\leftrightarrow$ゴーシュの相互遷移のエネルギー障壁が、\qty{15}{kJ mol^{-1}}程度であるとした場合、室温での遷移の頻度を見積もってください。
\end{enumerate}

\end{document}