\documentclass[a4paper,11pt]{jlreq}
% 基本とドライバ関連
\usepackage{graphicx}
\usepackage{xcolor}
\usepackage{makeidx}
\usepackage{ascmac}

% LuaTeX-ja設定
\usepackage{luatexja}% 日本語したい
\usepackage[haranoaji,no-math,deluxe,expert,nfssonly,match,scale=1.0]{luatexja-preset}
\renewcommand{\kanjifamilydefault}{\gtdefault}% 既定をゴシック体に
\usepackage{lltjext}

% 数式系基本
\usepackage{amsmath}
\usepackage{amsthm}
\usepackage{amssymb}
\usepackage{mathtools}
% \mathtoolsset{showonlyrefs=true}
\usepackage{derivative}
\usepackage[b]{esvect}
\usepackage{nicematrix}
\usepackage{siunitx}
\usepackage{bm}

% 画像関係
\usepackage{animate}
\usepackage{svg}
\usepackage{tikz}

%表関連
\usepackage{multirow}

% 自然科学用追加
% \usepackage{chemmacros}
% \usechemmodule{all}
% \selectchemgreekmapping{fontspec}
\usepackage{chemfig}
\setchemfig{atom sep=1.5em}
% \ifdraft{}{\setchemfig{bond join=true}}

% 数式フォント設定
\usepackage{anyfontsize}
\newcommand{\sfscale}{0.98}
\newcommand{\ttscale}{0.96}
% \usepackage[mathnoalias]{iwona}
% % \setmainfont{Iwona}
% \usepackage[scale=\sfscale]{roboto}
% \usepackage[scale=\ttscale]{roboto-mono}
% \usepackage{BOONDOX-uprscr}
% \usepackage{BOONDOX-ds}

% ページ設定
\usepackage{geometry}
\geometry{left=25truemm, right=25truemm, top=25truemm, bottom=25truemm}
% \pagestyle{empty}

% hyperref関連
\usepackage{bookmark}
\usepackage{xurl}
\hypersetup{unicode,bookmarksnumbered=true,colorlinks=true,final}

%%%%%%%%%%%%%%%%%%%%%%%
\graphicspath{{../figure/}{../../figure/}}

\usepackage{xr}
\externaldocument{./polymer_image_exam}
\externaldocument{./polymer_image_app}

\begin{document}
\section{高分子鎖のイメージ}
高分子鎖は、模式的には「非常に細くてとても長い紐が、くるくると丸まったもの」と表現され、「このような高分子鎖の形状に起因して低分子とは大きく異なった高分子に特有な振る舞いを示す」とされている。
高分子を使いこなして特徴ある機能を発現できる材料とするためには、高分子に特有の形状をきちんと理解することが必須となるわけです。

このような高分子のイメージについて、まず、ミクロな化学構造の結合状態をみることから始めて、少しずつ理解してみよう。

\subsection{細くて長い}

高分子の例として、メチレンを繰り返し単位とした連鎖からなる最も単純な構造であるポリエチレンを対象として考えてみよう。

\subsubsection{炭素結合に着目して}
その構造は、エチレンモノマー単位が重合により多数連結した鎖であり、それぞれのモノマー単位は\chemfig{C-C}結合でつながっている。
なお、\chemfig{C-C}結合の結合長は\qty{0.154}{nm}で、3つの炭素が形成する\chemfig{C-C-C}結合は約\ang{109.5}の結合角を有している
\footnote
{
炭素原子が正四面体の重心に来るようにおいた場合に、その四つの結合のそれぞれが正四面体の頂点となるような構造を取っていることから、この結合角は理解できる。
この結合角の具体的な導出については、\ref{sec: carbon_BA} に示した。
}。

\begin{figure}[htb]
	\centering
		\includesvg[width=.25\textwidth]{carbon.svg}
		\caption{炭素原子の立体構造}
		\label{fig:carbon}
\end{figure}

ここで、ポリマーの主鎖のモデルとして、まず、炭素が四個のアルキルであるブタンを考えてみよう。
真ん中の炭素結合(2位の炭素と 3位の炭素との間の結合:\chemfig{CH3-CH2-CH2-CH3})に対して、ニューマン投影式を考えよう。
エネルギーの低い状態として、それぞれのメチル基がトランスの位置
\footnote
{
トランス配座は、最初の三つの炭素の張る平面上で1位の炭素と4位の炭素が反対の位置(トランス)となる状態であり、ポテンシャル・エネルギーが最低となる。
}
と、ゴーシュの位置
\footnote
{
ゴーシュ配座は、トランスの位置からみて回転角$\phi = \pm$\qty{60}{\degree}に4位の炭素が位置するため、局所的にエネルギー極小となる状態となる。
この場合、1,4位の炭素の相互作用が存在するため、トランス状態よりはエネルギーは少し高いことになる。
}
になる状態を想定できることになる。
この時のエネルギー状態図も併せて、図\ref{fig:butane}に示した。

\begin{figure}[htb]
 \centering
	\includesvg[width=.8\textwidth]{butane.svg}
	\caption{ブタンの配座とエネルギー状態図}
	\label{fig:butane}
\end{figure}

ブタンのモデルをベースに考えると、ポリマー主鎖は、ブタンの両端のメチル基の位置からさらに結合が伸びているものとモデル化して考えることができる。
ここで、熱的にもっとも安定な状態(よりエネルギーの低いトランス状態)の極限として主鎖の炭素が全部トランスの配置にある状態を考えよう。
このとき、ポリマー鎖は1つの平面上でジグザグに伸びた形(平面ジグザグ構造)となる。

簡便のために、水素を無視して、\chemfig{C-C}結合(結合長は\qty{0.154}{nm})のみを考えてみよう。
このとき、ポリエチレンの直径は約 0.09 nm、また、ポリマー鎖中のモノマー単位の長さは約 0.25 nm と見積もることができる(章末問題 \ref{it:2-1})。
このようなモノマー単位が、平面ジグザグ構造の連鎖として直線的につながったと考えてみよう。
例えば、ポリエチレンの分子量が\num{100000}であった場合には、その長さは約 900 nm となる
\footnote
{
この時、縦横比を表すアスペクトレシオを考えると、10000ということになる。
}(章末問題 \ref{it:2-2})。

もっと手に取って扱えるような大きさで考えてみよう。
このヒモを直径 5 mm のチョークを使って横一本の線で黒板に書いたとすると、その長さは約 50 m (普通の黒板の 10 枚分程度)ということになる。
いかに高分子が細長い形状をしているかが想像できるでしょう。
 


\subsection{くるくる丸まる}

高分子をピンと引き延ばすと、非常にアスペクト比の高い細長いものとなることを、上に示した。
しかしながら、自然な状態での実際の高分子は、決してこのような伸び切り構造を取っているわけではない。
ぐるぐるに丸まった、ランダムコイルと呼ばれる状態になっていることが知られている。
このことを、統計力学的な考え方で確かめてみよう。

統計力学によれば、エネルギーの異なる2つの状態(エネルギー差 $\Delta E$)の出現確率の比 $r(\Delta E)$ は、ボルツマン因子を用いて、下式のように表すことができる
\footnote
{
ボルツマン因子とは、温度 $T$ の熱平衡状態にある系において、エネルギー $E_i$ である状態 $i$ が出現する相対確率 $P(E_i)$ を、下式のように定める重み因子となる。
ボルツマン因子は、通常カノニカル分布によって記述される系を議論する際に用いられる。
\begin{align*}
P(E_i) = \dfrac{1}{Z} \exp(-\beta E_i)
\end{align*}
ここで、$Z=\sum_i \exp(-\beta E_i)$ は、$\sum_i P_i = 1$ とするための規格化因子であり、分配関数、あるいは、状態和と呼ばれる。

}(章末問題 \ref{it:2-3})。
\begin{equation}
r(\Delta E) = \exp(-\beta \Delta E)=\exp \left( -\dfrac{\Delta E}{k_B T} \right)
\label{eq:ratio}
\end{equation}

ただし、$\beta=\dfrac{1}{k_B T}$、$k_B$ はボルツマン定数($k_B =$\qty{1.38E-23}{J K^{-1}})、$T$ は絶対温度です。

このことを利用すれば、前述のトランスとゴーシュ状態とのポテンシャル・エネルギー差 $\Delta \varepsilon$ に基づいて、熱平衡状態でのそれぞれの状態の出現確率の比を求めることができる。
トランス状態が連続する限り、ポリマー鎖は平面ジグザグ構造により直線的に伸びていくが、ゴーシュが出現することで鎖の曲りが生じる。
一般に、このエネルギー差$\Delta \varepsilon$は数\unit{k J mol^{-1}}程度となっている。
したがって、数個程度のモノマー単位の連鎖ごとにゴーシュ状態が出現し、ポリマー主鎖が曲ることになる(章末問題 \ref{it:2-4}、\ref{it:2-5})。

前述のようにポリマー鎖は非常に細長いものであるから、10個以下の連鎖で少し曲がるような程度の曲りであっても、全体でみればくるくると丸まったものになっていることが理解できる。

\subsection{ぐにゅぐにゅと蠢く}

トランス状態からゴーシュ状態へと遷移する際のエネルギー障壁($\Delta E$)を用いた考察から、この遷移の起こる頻度を見積もることもできる。
分光学的な測定から、ブタンにおける \chemfig{C-C}結合周りの回転振動数は、\qty{10E12}{sec^{-1}}のオーダーで生じることが知られている。

トランス $\leftrightarrow$ ゴーシュの相互遷移のエネルギー障壁は\numrange{10}{20}\unit{kJ mol^{-1}}程度であるので、ボルツマン因子を用いて評価すると、室温での遷移は\qty{1E9}{sec^{-1}}程度のオーダーで生じるものと見積もることができる(章末問題 \ref{it:2-6})。

実際の高分子においては、主鎖周りの影響を受けるためこの回転運動の振動数は低いものとなり、また、エネルギー障壁も大きくなると考えられる。
しかしながら、それでも十分に高い頻度でこのような回転異性化によるコンフォメーションの変化が生じるであろうということが想像できる。
このような変化が、高分子のミクロブラウン運動と呼ばれる運動モードの起源の主要なひとつとなっている。

\end{document}