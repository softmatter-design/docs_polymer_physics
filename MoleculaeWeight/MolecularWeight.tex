\documentclass[a4paper,11pt]{ltjsarticle}
% 基本とドライバ関連
\usepackage{graphicx}
\usepackage{xcolor}
\usepackage{makeidx}
\usepackage{ascmac}

% LuaTeX-ja設定
\usepackage{luatexja}% 日本語したい
\usepackage[haranoaji,no-math,deluxe,expert,nfssonly,match,scale=1.0]{luatexja-preset}
\renewcommand{\kanjifamilydefault}{\gtdefault}% 既定をゴシック体に
\usepackage{lltjext}

% 数式系基本
\usepackage{amsmath}
\usepackage{amsthm}
\usepackage{amssymb}
\usepackage{mathtools}
% \mathtoolsset{showonlyrefs=true}
\usepackage{derivative}
\usepackage[b]{esvect}
\usepackage{nicematrix}
\usepackage{siunitx}
\usepackage{bm}

% 画像関係
\usepackage{animate}
\usepackage{svg}
\usepackage{tikz}

%表関連
\usepackage{multirow}

% 自然科学用追加
% \usepackage{chemmacros}
% \usechemmodule{all}
% \selectchemgreekmapping{fontspec}
\usepackage{chemfig}
\setchemfig{atom sep=1.5em}
% \ifdraft{}{\setchemfig{bond join=true}}

% 数式フォント設定
\usepackage{anyfontsize}
\newcommand{\sfscale}{0.98}
\newcommand{\ttscale}{0.96}
% \usepackage[mathnoalias]{iwona}
% % \setmainfont{Iwona}
% \usepackage[scale=\sfscale]{roboto}
% \usepackage[scale=\ttscale]{roboto-mono}
% \usepackage{BOONDOX-uprscr}
% \usepackage{BOONDOX-ds}

% ページ設定
\usepackage{geometry}
\geometry{left=25truemm, right=25truemm, top=25truemm, bottom=25truemm}
% \pagestyle{empty}

% hyperref関連
\usepackage{bookmark}
\usepackage{xurl}
\hypersetup{unicode,bookmarksnumbered=true,colorlinks=true,final}

%%%%%%%%%%%%%%%%%%%%%%%
\graphicspath{{../figure/}{../../figure/}}

\begin{document}
\section{高分子の分子量}

高分子は多数のモノマーが連結したものであり、一つの分子が非常に大きな分子量を有していることになる。
この分子量という「数」を明確に定義することで、「長さ」あるいは「エントロピー」のような物理量を導出でき、材料設計に役立つ情報とすることができる。

高分子を材料として使いこなしていくためには、使用条件下でどのぐらいの大きさを占めているのかを見積もることが必要になる場合が多い。
例えば、ミクロ相分離のような構造において界面近傍での挙動を記述する際には、ポリマー鎖の変形がどのように生じているのかを考察するためには、バルク状態での鎖の大きさを知る必要がある。
また、高分子を希釈して使用する場合にも、それぞれのポリマー鎖がどの程度重なり合うのかに応じて巨視的な粘度等の挙動が大きく変化するので、希釈条件下での大きさの情報は重要である。
前章で示した慣性半径のような特徴長さは、分子量 $N$ に対して以下の関係があり、
\begin{align*}
\begin{cases}
\text{高分子の大きさを表す長さ} \propto N^{1/2} \\[8pt]
\text{高分子の体積} \propto N^{3/2}
\end{cases}
\end{align*}
高分子の大きさの変化の傾向を見積もることができるようになる。


また、分子量の増加に伴い分子間あるいは分子内における相互作用点が多くなり、マクロな物性値は変化する。
さらに、連結数(重合度)の増加とともに、それぞれの連結において取りうる配座の数は、指数関数的に増加する。
このことは、統計力学的なエントロピーの増加に他ならない。

したがって、材料としての機能を設計するためには、高分子の分子量を定義することは重要である。
以下に、高分子の大きさを見積もるために必要となる分子量の決定方法について、簡単に記述しよう。


\subsection{分子量の表し方}

合成高分子は、その合成法によって規定される統計的偶然性のために、分子量の分布を生じてしまう。
したがって、高分子の分子量は、低分子化合物とは異なり、分子量の平均すなわち平均分子量で表さざるを得ないことになる。
その平均の仕方には、各種考えられるが、一般に以下の二つが多用されている。

なお、ここでは、分子量が $M_i$ の高分子鎖が $n_i$ 本含まれている系(ただし、インデックス変数 $i$ は、 $i=0,1,\cdots$)を対象として考えている
\footnote
{
ここで、用いている $i$ というインデックスは、系中に存在するすべてのポリマー鎖を一本ずつ調べてヒストグラムで表した場合に、ヒストグラム中の何番目の括りに入っているのかということを表している。
}。


このとき、$\dfrac{n_i}{\sum_i n_i}$ は、すべての分画の本数を足し合わせたもので $i$ 成分の本数を割ったものであるので、$i$ 成分の数分率を表し、また、$w_i=\dfrac{n_i M_i}{\sum_i n_i M_i}$ は、$i$ 成分の重量分率を表していることになる。

\begin{itemize}
\item
数平均分子量

これは、以下の式で表されるように、数(分子の本数)平均で分子量を表したものであり、分子の個数を評価することにより決定できる(章末問題 \ref{it:4-1})。

この表現の分子量は、例えば、末端基定量法や浸透圧法により測定できる。
\begin{align}
{\bar M}_n 
	&= \dfrac{\sum_i n_i M_i}{\sum_i n_i} \notag \\[6pt]
%	= \dfrac{n_0 M_0}{\displaystyle \sum_i n_i} + \dfrac{n_1 M_1}{\displaystyle \sum_i n_i}
	&= \dfrac{1}{\sum_i \left(\dfrac{w_i}{M_i} \right)}
\label{eq:Mn}
\end{align}

\item
重量平均分子量

任意のインデックス $i$ で指定されるポリマーの重量 $n_i M_i$ の重量分率で分子量を平均化したものであり、光散乱法や沈降平衡法により測定できる。

これは、以下のような表式で表される(章末問題 \ref{it:4-2})。
\begin{align}
{\bar M}_w 
	&= \dfrac{\sum_i n_i M_i^2}{\sum_i n_i M_i}\notag \\[6pt]
%	&= \dfrac{n_0 M_0^2}{\displaystyle \sum_i n_i M_i} + \dfrac{n_1 M_1^2}{\displaystyle \sum_i n_i M_i} + \cdots \notag \\[6pt]
%	&= M_0 \dfrac{n_0 M_0}{\displaystyle \sum_i n_i M_i} + M_1 \dfrac{n_1 M_1}{\displaystyle \sum_i n_i M_i} + \cdots \notag \\[6pt]
%	&= M_0 w_0 + M_1 w_1 + \cdots \notag \\[6pt]
	&= \displaystyle \sum_i w_i M_i
\label{eq:Mw}
\end{align}

\end{itemize}

ここまでは、ヒストグラムとして取り扱えるように、離散的(整数のようなとびとびの値)に分画した議論を行ってきた。

通常、我々が扱う高分子の分子量は非常に大きいものであるので、分子量 $M$ を連続的な量であるとみなしても問題ない。
任意の試料において、分子量 $M$ と $M+dM$ の間にある分子の数分率を$f_n(M)dM$、重量分率を$f_w(M)dM$ と置くと、$f_n(M)$ および $f_w(M)$ は、それぞれ、数微分分布関数および重量微分分布関数となる。
このとき、数平均分子量 $M_n$、および、重量平均分子量 $M_w$ は、それぞれ、以下のように積分を用いた連続的な表現に書き直すことができる
\footnote
{
統計では、確率変数 $X$ の期待値(平均値)$E(x)$ は、確率密度関数 $f(x)$ を用いて、以下のように書くことができる。
\begin{align*}
E(x) = \displaystyle \int x f(x) dx
\end{align*}
}。
\begin{align}
\begin{cases}
M_n = \displaystyle\int_0^{\infty} M f_n (M)dM \\[10pt]
M_w = \displaystyle\int_0^{\infty} M f_w (M)dM
\end{cases}
% = \dfrac{1}{\displaystyle\int_0^{\infty} \dfrac{1}{M} f_w (M)dM}
\end{align}

\subsection{分子量分布}

前述のように、合成高分子は分子量の分布を生じてしまうのであった。
その分布の度合いは、分布関数の標準偏差により評価することができる(章末問題 \ref{it:4-3})。

高分子鎖の本数に関する分布関数の標準偏差 $s_n$ は以下のように定義され
\footnote
{
一般に、標準偏差は、平均値 $\mu$ を用いて、以下のように定義されている。
\begin{align*}
s^2
	&= \displaystyle \int(x - \mu)^2 f(x) dx
\end{align*}
}、
\begin{align}
s_n^2
	&= \displaystyle \int_0^{\infty}(M - \bar{M_n})^2 f_n(M) dM
\end{align}
ここから、以下の式が導出できる
\footnote
{
この導出過程は、\ref{sec:MwMn} に示した。
}。
\begin{align}
\dfrac{s_n}{\bar{M}_n} = \left(\dfrac{\bar{M}_w}{\bar{M}_n}-1\right)^{1/2}
\end{align}

標準偏差 $s_n$ が大きいということが分布関数の幅が広い、すなわち、不均一の度合いが大きいことを表しているのであった。
上式において、$\dfrac{\bar{M}_w}{\bar{M}_n}$ が 1 よりも大きいほど右辺が大きくなり、一方、左辺は $\bar{M}_n$ は定数であるので、分子量の不均一度が大きいことになる。
これが、$\dfrac{\bar{M}_w}{\bar{M}_n}$ が分子量分布の幅を見積もる指標となる理由の説明である。
 
GPC(Gel Permiation Chromatography)法は、多孔性の高分子ゲルからなるカラムに高分子溶液を流し、分子量の大きい順に溶出してきた高分子溶液の濃度を測定する方法である。
単分散高分子試料を用いて校正曲線を作成することで、試験試料中の分子量の異なるポリマーの重量分率の分布を決定することができる。
上述したように、重量分率 $w_i$ の分布関数が求まれば、数平均分子量 $\bar{M}_n$ および重量平均分子量 $\bar{M}_w$ を求められるのであったから、GPC法により、それぞれの平均分子量および分子量分布の不均一度が同時に測定できることになる。



\subsection{章末問題}

	\begin{enumerate}
	\item
	(和のとり方の確認)\\
		\vspace{-5mm}
		\begin{enumerate}
		\item
		\label{it:4-1}
		(\ref{eq:Mn}) 式の二行目への展開を実際にやってみてください。\\
		(ヒント)\\
		$\sum$ で表現されている和を実際に開いてみて、$w_i$ とよく見比べてみてください。

		\item
		\label{it:4-2}
		(\ref{eq:Mw}) 式の二行目への展開を実際にやってみてください。\\
		(ヒント)\\
		上のクイズと同様に、$\sum$ で表現されている和を実際に開いてみてください。
		\end{enumerate}
	\item
	(分子量分布の指標)\\
	\label{it:4-3}
	$\dfrac{\bar{M}_w}{\bar{M}_n}$ が、分子量の広がりを評価する指標となる理由について、説明してください。

	\end{enumerate}

\end{document}